% !TEX TS-program = pdflatex
% !TEX encoding = UTF-8 Unicode

% This is a simple template for a LaTeX document using the "article" class.
% See "book", "report", "letter" for other types of document.

\documentclass[11pt]{article} % use larger type; default would be 10pt

\usepackage[utf8]{inputenc} % set input encoding (not needed with XeLaTeX)

%%% Examples of Article customizations
% These packages are optional, depending whether you want the features they provide.
% See the LaTeX Companion or other references for full information.

%%% PAGE DIMENSIONS
\usepackage{geometry} % to change the page dimensions
\geometry{a4paper} % or letterpaper (US) or a5paper or....
% \geometry{margin=2in} % for example, change the margins to 2 inches all round
% \geometry{landscape} % set up the page for landscape
%   read geometry.pdf for detailed page layout information

\usepackage{graphicx} % support the \includegraphics command and options

% \usepackage[parfill]{parskip} % Activate to begin paragraphs with an empty line rather than an indent

%%% PACKAGES
\usepackage{booktabs} % for much better looking tables
\usepackage{array} % for better arrays (eg matrices) in maths
\usepackage{paralist} % very flexible & customisable lists (eg. enumerate/itemize, etc.)
\usepackage{verbatim} % adds environment for commenting out blocks of text & for better verbatim
\usepackage{subfig} % make it possible to include more than one captioned figure/table in a single float
% These packages are all incorporated in the memoir class to one degree or another...
\usepackage{mathtools}
\usepackage{algpseudocode}



%%% HEADERS & FOOTERS
\usepackage{fancyhdr} % This should be set AFTER setting up the page geometry
\pagestyle{fancy} % options: empty , plain , fancy
\renewcommand{\headrulewidth}{0pt} % customise the layout...
\lhead{}\chead{}\rhead{}
\lfoot{}\cfoot{\thepage}\rfoot{}

%%% SECTION TITLE APPEARANCE
\usepackage{sectsty}
\allsectionsfont{\sffamily\mdseries\upshape} % (See the fntguide.pdf for font help)
% (This matches ConTeXt defaults)

%%% ToC (table of contents) APPEARANCE
\usepackage[nottoc,notlof,notlot]{tocbibind} % Put the bibliography in the ToC
\usepackage[titles,subfigure]{tocloft} % Alter the style of the Table of Contents
\renewcommand{\cftsecfont}{\rmfamily\mdseries\upshape}
\renewcommand{\cftsecpagefont}{\rmfamily\mdseries\upshape} % No bold!

%%% END Article customizations

%%% The "real" document content comes below...

\title{Applied Mathematics Home Work 3}
\author{Emre Ozan Alkan}
%\date{} % Activate to display a given date or no date (if empty),
         % otherwise the current date is printed 

\begin{document}
\maketitle

\section{Problem 1}

For the first day, we have 750 people purchased newspaper and 250 people did not. I constructed v vector for day zero as

$$v_{0} = 
  \begin{pmatrix} 750 \\ 250 \end{pmatrix}
$$
My Markov matrix, where each column's sum is 1, is created with the given values

$$ M = 
 \begin{bmatrix}
	0.7 & 0.2 \\ 0.3 & 0.8
 \end{bmatrix}
$$

\subsection{If a person purchased a paper today, how likely is he to purchase a paper on Day 2? Day 3? Day n?}

Calculating  probability of one person, we should change our v vector as;

$$v_{0} = 
  \begin{pmatrix} 1 \\ 0 \end{pmatrix}
$$
to represent one person. For calculating, we need to use our Markov Matrix. Then the calculations for Day 2 , Day 3 and Day n is respectively;

$$Day\ 2, v_{2} = 
   \begin{bmatrix}
	0.7 & 0.2 \\ 0.3 & 0.8
 \end{bmatrix}     \begin{bmatrix}
	0.7 & 0.2 \\ 0.3 & 0.8
 \end{bmatrix}  \begin{pmatrix} 1 \\ 0 \end{pmatrix} = \%55\ likehood\ to\ buy
$$

$$Day\ 3, v_{3} = 
   \begin{bmatrix}
	0.7 & 0.2 \\ 0.3 & 0.8
 \end{bmatrix}        \begin{bmatrix}
	0.7 & 0.2 \\ 0.3 & 0.8
 \end{bmatrix}     \begin{bmatrix}
	0.7 & 0.2 \\ 0.3 & 0.8
 \end{bmatrix}  \begin{pmatrix} 1 \\ 0 \end{pmatrix} = \%47.5\ likehood\ to\ buy
$$

$$Day\ n, v_{n} = 
   \begin{bmatrix}
	0.7 & 0.2 \\ 0.3 & 0.8
 \end{bmatrix} ^ {n} \begin{pmatrix} 1 \\ 0 \end{pmatrix}
$$


\subsection{What sales figures can The Computer Visionist expect on Day 2? Day 3? Day n?}

To calculate sales figures for whole people, we should consider our v vector in Day zero as;

$$v_{0} = 
  \begin{pmatrix} 750 \\ 250 \end{pmatrix}
$$

$$Day\ 2, v_{2} = 
   \begin{bmatrix}
	0.7 & 0.2 \\ 0.3 & 0.8
 \end{bmatrix} ^ {2} \begin{pmatrix} 750 \\ 250 \end{pmatrix}
 = \begin{pmatrix} 487.5 \\ 512.5 \end{pmatrix} = 487.5\ sales
$$

$$Day\ 3, v_{3} = 
   \begin{bmatrix}
	0.7 & 0.2 \\ 0.3 & 0.8
 \end{bmatrix} ^ {3} \begin{pmatrix} 750 \\ 250 \end{pmatrix}
 = \begin{pmatrix} 443.75 \\ 556.25 \end{pmatrix} = 443.75\ sales
$$

$$Day\ n, v_{3n} = 
   \begin{bmatrix}
	0.7 & 0.2 \\ 0.3 & 0.8
 \end{bmatrix} ^ {n} \begin{pmatrix} 750 \\ 250 \end{pmatrix}
$$



\subsection{Will the sales figures fluctuate a great deal from day to day, or are they likely to become stable eventually ?}

Here is our Markov Matrix:

$$ M = 
 \begin{bmatrix}
	0.7 & 0.2 \\ 0.3 & 0.8
 \end{bmatrix}
$$

For n =  1 to infinity, $ M^{n} $ goes to be stable.

$$ M^{1} = 
 \begin{bmatrix}
	0.7 & 0.2 \\ 0.3 & 0.8
 \end{bmatrix}
$$

$$ M^{2} = 
 \begin{bmatrix}
	0.55 & 0.3 \\ 0.45 & 0.7
 \end{bmatrix}
$$

$$\dots$$

$$ M^{10} = 
 \begin{bmatrix}
	0.4006 & 0.3996 \\ 0.5994 & 0.6004
 \end{bmatrix}
$$

$$\dots$$

$$ M^{14} = 
 \begin{bmatrix}
	0.4 & 0.4 \\ 0.6 & 0.6
 \end{bmatrix}
$$

$$\dots$$

$$ M^{20} = 
 \begin{bmatrix}
	0.4 & 0.4 \\ 0.6 & 0.6
 \end{bmatrix}
$$

$$\dots$$

After 14th day, sales figure become stable.






\section{Problem 2}


\subsection{Summary of the ranking method}


\subsection{How eigenvalue problem solved}


\subsection{5x5 Markov matrix, algorithm application}


\subsection{Matlab Function}


\subsection{Comments}


\end{document}

\documentclass[10pt,a4paper,titlepage]{report}
\usepackage[utf8]{inputenc}
\usepackage{amsmath}
\usepackage{amsfonts}
\usepackage{amssymb}
\author{Emre Ozan Alkan \\ \{emreozanalkan@gmail.com\}}
\title{Marie Jean Antoine Nicolas de Caritat Condorcet}
\begin{document}
\maketitle

\begin{center}
\title{Marie Jean Antoine Nicolas de Caritat Condorcet}\\
\author{Emre Ozan Alkan \\ \{emreozanalkan@gmail.com\}}
\end{center}

Marie Jean Antoine Nicolas de Caritat(born September 17, 1743, Ribemont, France—died March 29, 1794, Bourg-la-Reine), marquis de Condorcet, known as Nicolas de Condorcet, was an arictocrat, a mathematician, an official of the Academy of Sciences, a French philosopher and early polical scientist whose name is given to our campus of University of Burgundy in Le Creusot.


He is one of the philosophers of the Enlightenment and advocate of educational reforms. Furthermore, he was one of the revolutionary inventer of the ideas of progress. Unlike many others, he advocated a liberal economy, free and equal public education and equal right for humanity.


Condorcet was born in Ribermont, he became fatherless at young age and raised by his religous mother.


He had early political career in 1774 by being Inspector General of tehe Monnaie de Paris.


In 1785, he made publication in probability and found method called with his lastname: Condorcet method and Condorcet's paradox. He was one of the first people who applied math in social sciences.

While working on Sketch for a Historical Picture of the Progress of the Human Spirit, he was hiding from government. 

He never lost his belief on progression of human kind and mind which requires respect and place in heart of all humans.

\end{document}
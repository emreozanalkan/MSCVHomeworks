\documentclass{article}

\usepackage[utf8]{inputenc}
\usepackage{amsmath}
\usepackage{amssymb}
\usepackage{anysize}
\usepackage{color}
\usepackage{xcolor}
\usepackage{hyperref}
\usepackage{float}

\usepackage{listings}
\lstset{
	language=C++,                	% choose the language of the code
	basicstyle=\footnotesize,       % the size of the fonts that are used for the code
	numbers= left,                 	% where to put the line-numbers
	numberstyle=\footnotesize,      % the size of the fonts that are used for the line-numbers
	stepnumber=1,                   % the step between two line-numbers. If it is 1 each line will be numbered
	numbersep=5pt,                  % how far the line-numbers are from the code
	backgroundcolor=\color{white},  % choose the background color. You must add \usepackage{color}
	showspaces=false,               % show spaces adding particular underscores
	showstringspaces=false,         % underline spaces within strings
	showtabs=false,                 % show tabs within strings adding particular underscores
	frame=single,           		% adds a frame around the code
	tabsize=2,          			% sets default tabsize to 2 spaces
	captionpos=t,          			% sets the caption-position to bottom (t=top, b=bottom)
	breaklines=true,        		% sets automatic line breaking
	breakatwhitespace=false,    	% sets if automatic breaks should only happen at whitespace
	escapeinside={\%*}{*)}          % if you want to add a comment within your code
}



\usepackage{caption}
\DeclareCaptionFont{white}{\color{white}}
\DeclareCaptionFormat{listing}{\colorbox{gray}{\parbox[c]{\textwidth}{#1#2#3}}}
\captionsetup[lstlisting]{format=listing,labelfont=white,textfont=white}

\setlength\parindent{0pt}
\setlength{\parskip}{10pt}

\marginsize{3cm}{2cm}{2cm}{2cm}

\title{Autonomous Robotics\\
		Autocalibration Report}
\author{Emre Ozan Alkan\\
		\{emreozanalkan@gmail.com\}\\
		MSCV-5}
\date{\today}

\begin{document}
\maketitle

\section{Introduction}

Camera calibration with known patterns or objects with know euclidean structures are very well known and used techniques. However these methods are not available without priori assumptions to calibrate and get intrinsic parameters. Hence, we need to use autocalibration techniques that doesn't rely on euclidean structures for calibration. 


\section{Methods}
	We have been given approximate intrinsic parameter of the camera, fundemental matrixes between the images, projection matrix between the images. We employ 3 different method below to more accurately approximate intrinsic parameters.
	
	\subsection{Mendonça and Cipolla}
		Basic method based on the exploitation of the rigidity constraint. We design cost function, which takes intrinsic parameters and fundemental matrices as parameters, and returns positive value related to difrence between two non-zero singular value of the essential matrix.		

	\begin{table}[H]
	\centering
    \begin{tabular}{lll}
    800.018875334029 & -0.00111165444322615 & 255.996941864705 \\
    0                & 800.019278207122     & 256.001480852401 \\
    0                & 0                    & 1                \\
    \end{tabular}
    \caption {Mendonca Cipolla Intrinsic Parameter Approximation}
	\end{table}
		
		\begin{lstlisting}[label=mendonca-and-cipolla-cost, caption=Mendonca and Cipolla Cost]
function [ C ] = mendoncaCipollaCost( X, Fs )
%MENDONCACIPOLLACOST Summary of this function goes here
%   Detailed explanation goes here

% X(1) = alfa_u
% X(2) = alfa_v
% X(3) = gamma
% X(4) = u_0
% X(5) = v_0

% Hint: For Mandonca&Cipolla, consider all the weight w_ij = 1

% SVD = and get S, first 2 sigma
% S = [sigma_1   0         0;
%      0         sigma_2   0;
%      0         0         0];
%
% Essential = A' * Fs * A;

A = [X(1)   X(3)   X(4);
    0       X(2)   X(5);
    0       0      1];

E = zeros(3, 3, 10, 10);

C = zeros(1, 10);

for ii = 1 : 10
    
    for jj = 1 : 10
        
        if ii == jj
            continue;
        end
        
        E(:, :, ii, jj) = A' * Fs(:, :, ii, jj) * A;

        [~, S, ~] = svd(E(:, :, ii, jj));
        
        C(ii) =  C(ii) + ((S(1, 1) - S(2, 2)) / S(2, 2));
        
    end
    
end

C = sum(C(:));


end
\end{lstlisting}  

\begin{lstlisting}[label=mendonca-and-cipolla-test, caption=Mendonca and Cipolla Test]
load('data.mat');

X = zeros(5, 1);

X(1) = A(1, 1);
X(2) = A(2, 2);
X(3) = A(1, 2);
X(4) = A(1, 3);
X(5) = A(2, 3);

costFunc =  @(X) mendoncaCipollaCost(X, Fs);

[xHat, resnorm] = lsqnonlin(costFunc, X);

% xHat(1) = alfa_u
% xHat(2) = alfa_v
% xHat(3) = gamma
% xHat(4) = u_0
% xHat(5) = v_0
A2 = [xHat(1)   xHat(3)   xHat(4);
     0         xHat(2)   xHat(5);
     0         0         1];
 
\end{lstlisting}  


	\subsubsection{Kruppa Equations}
		Kruppa equations are obtatining intrinsic parameters of the camera using polynomial equations, with a minimum of three displacements. 
		I had hard time to achieve good results without passing the tolerance, and some more parameter values for lsqnonlin function. 
		
			\begin{table}[H]
	\centering
    \begin{tabular}{lll}
    800.008699400211 & -0.000841395721280208 & 256.000110817581 \\
    0                & 800.00842085238    & 256.000185844728 \\
    0                & 0                    & 1                \\
    \end{tabular}
    \caption {Kruppa Equations Intrinsic Parameter Approximation}
	\end{table}
		
		\begin{lstlisting}[label=kruppa-equations-cost, caption=Kruppa Equations Cost]
function [ C ] = kruppaCost( X, Fs )
%KRUPPACOST Summary of this function goes here
%   Detailed explanation goes here

% X(1) = alfa_u
% X(2) = alfa_v
% X(3) = gamma
% X(4) = u_0
% X(5) = v_0

A = [X(1) X(3) X(4);
    0    X(2) X(5);
    0    0    1];

w_inv = A * A';

E = zeros(3, 3, 10, 10);

%C = zeros(1, 10);
C = 0;

for ii = 1 : 10
    
    for jj = 1 : 10
        
        if ii == jj
            continue;
        end
        
        %         E(:, :, ii, jj) = A' * Fs(:, :, ii, jj) * A;
        %         
        %         %[~, S, ~] = svd(E(:, :, ii, jj));
        %         C(ii) = C(ii) + norm(((Fs(:, :, ii, jj) * w_inv * Fs(:, :, ii, jj)') / norm(Fs(:, :, ii, jj) * w_inv * Fs(:, :, ii, jj)', 'fro')) - ...
        %         ((E(:, :, jj, ii) * w_inv * E(:, :, jj, ii)') / norm(E(:, :, jj, ii) * w_inv * E(:, :, jj, ii)' ,'fro') ), 'fro');
        %         %C(ii) =  C(ii) + ((S(1, 1) - S(2, 2)) / S(2, 2));
        
        
        
        % Abinash's way instead of longer equation
        currFS = Fs(:, :, ii, jj);
        ce = null(currFS');
        F = currFS * w_inv * currFS';
        F = F / norm(F, 'fro');
        currE = [0 -ce(3) ce(2); ce(3) 0 -ce(1); -ce(2) ce(1) 0];
        E = currE * w_inv * currE';
        E = E / norm(E, 'fro');
        tempC = F - E;
        C = C + norm(tempC,'fro');
        % Abinash's way
        
    end
    
end

end
\end{lstlisting}  			

\begin{lstlisting}[label=kruppa-equations-test, caption=Kruppa Equations Test]
load('data.mat');

X = zeros(5, 1);

X(1) = A(1, 1);
X(2) = A(2, 2);
X(3) = A(1, 2);
X(4) = A(1, 3);
X(5) = A(2, 3);

costFunc =  @(Y) kruppaCost(Y, Fs);

options = optimset('Algorithm','levenberg-marquardt','MaxFunEvals',10^50,'TolFun',10^-100,'TolX',10^-100);
[xHat, resnorm] = lsqnonlin(costFunc, X, [], [], options);

A3 = [xHat(1)   xHat(3)   xHat(4);
     0         xHat(2)   xHat(5);
     0         0         1];
\end{lstlisting}  

	\subsection{Dual Absolute Quadric}

	We represent euclidean scene structure formulated in terms of the absolute quadric - the singular dual 3D quadric giving teh Euclidean dot-product between plane normals.
	
	\begin{lstlisting}[label=dual-absolute-quadric, caption=Dual Absolute Quadric]
load('data.mat');

% A = [800 0 256; 0 800 256; 0 0 1]; % Correct one

w = A * A';
nx = sym('nx', 'real');
ny = sym('ny', 'real');
nz = sym('nz', 'real');
l2 = sym('l2', 'real');

n = [nx; ny; nz]; % Normal of pi inf

Q = [w, (w * n); (n' * w), (n' * w * n)]; % Dual Absolute Quadric

M2 = PPM(:, :, 2);

m2 = M2 * Q * M2';

sol = solve(m2(1, 1) == (l2 * w(1, 1)), ...
            m2(2, 2) == (l2 * w(2, 2)), ...
            m2(3, 3) == (l2 * w(3, 3)), ...
            m2(1, 3) == (l2 * w(1, 3)));

display('sol.nx:');
display(double(sol.nx));
display('sol.ny:');
display(double(sol.ny));
display('sol.nz:');
display(double(sol.nz));
display('sol.l2:');
display(double(sol.l2));
\end{lstlisting}  






\section{References}

\begin{enumerate}
	\item Abinash Pant
	\item \url{http://en.wikipedia.org/wiki/Camera_auto-calibration}
	\item \url{http://homepages.inf.ed.ac.uk/rbf/CVonline/LOCAL_COPIES/FUSIELLO3/node4.html#SECTION00043000000000000000}
	\item \url{http://homepages.inf.ed.ac.uk/cgi/rbf/CVONLINE/entries.pl?TAG1325}
	\item \url{http://hal.inria.fr/docs/00/54/83/45/PDF/Triggs-cvpr97.pdf}
\end{enumerate}


\end{document}







































